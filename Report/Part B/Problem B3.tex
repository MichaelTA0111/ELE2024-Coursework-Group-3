\bigskip
\subsection{Problem B3}\label{sec:qb3}\hfill

\begin{figure}[ht]
    \centering
    \begin{minipage}{0.49\textwidth}
        \centering
        \includesvg[width=1\linewidth]{impulse_response.svg}
        \caption{Impulse response of the transfer function for the linear system.}
        \label{fig:impulse_response}
    \end{minipage}
    \begin{minipage}{0.49\textwidth}
        \centering
        \includesvg[width=1\linewidth]{step_response.svg}
        \caption{Step response of the transfer function for the linear system.}
        \label{fig:step_response}
    \end{minipage}
\end{figure}

\medskip
\noindent From Figure \ref{fig:impulse_response}, it can be seen that when a Dirac pulse is applied to the system, the ball will oscillate about the equilibrium position, i.e. move up and down the slope, with decreasing amplitude every oscillation. After approximately 0.6 s, the ball has stopped oscillating and has returned to its equilibrium position.
From Figure \ref{fig:step_response}, it can be seen that when a constant force is applied to the system, the ball will oscillate about a new equilibrium position approximately 9.5 mm from the original equilibrium position. After approximately 0.6 s, the ball has stopped at the new equilibrium position.