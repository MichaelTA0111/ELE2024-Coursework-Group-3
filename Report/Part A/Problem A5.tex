\bigskip
\subsection{Problem A5}\label{sec:qa5}\hfill

\medskip
\noindent Apply the Laplace Transform theorem to equation \eqref{a4:linear_system}, with all the initial conditions equal to zero:
\begin{align}
    &\lap \{\dot{\bar{x}}_1 (t) \} = \lap \{\bar{x}_2 (t) \} \cr
    \Longrightarrow &s \bar{X}_1 (s) - \bar{x}_1 (0^+) = \bar{X}_2 (s) \cr
    &s \bar{X}_1 (s) = \bar{X}_2 (s) \label{a5:laplace_x1_dot} \\ \vspace{1cm}
    &\lap \{ \dot{\bar{x}}_2 (t) \} = \lap \{D \bar{I} (t) + F \bar{x}_1 (t) - H \bar{x}_2 (t) \} \cr
    % \Longrightarrow &\lap \{ \dot{\bar{x}}_2 \} = D \lap \{ \bar{I}(t) \} + F \lap \{\bar{x}_1 \} - H \lap \{ \bar{x}_2 \} \cr
    \Longrightarrow &s \bar{X}_2 (s) - \bar{x}_2 (0^+) = D \bar{I}(s) + F \bar{X}_1 (s) - H \bar{X}_2 (s) \cr
    &s \bar{X}_2 (s) = D \bar{I}(s) + F \bar{X}_1 (s) - H \bar{X}_2 (s) \label{a5:laplace_x2_dot} \\ \vspace{1cm}
    &\lap \{ \dot{\bar{I}} (t)\} = \lap \{ N \bar{V} (t) - P \bar{I} (t) \} \cr
    \Longrightarrow &s \bar{I} (s) - \bar{I} (0^+) = N \bar{V} (s) - P \bar{I} (s) \cr
    &s \bar{I} (s) = N \bar{V} (s) - P \bar{I} (s) \cr
    &s \bar{I} (s) = N \bar{V} (s) - P \bar{I} (s) \cr
    &s \bar{I} (s) + P \bar{I} (s) = N \bar{V} (s) \cr
    &\bar{I} (s) = \frac{N \bar{V}(s)}{\left( s + P \right)} \label{a5:laplace_i_dot_0}
\end{align}
\noindent Substitute equation \eqref{a5:laplace_x1_dot} into equation \eqref{a5:laplace_x2_dot}:
\begin{align} \label{a5:laplace_i_dot_1}
    &s \bar{X}_2 (s) = D \bar{I} (s) + F \bar{X}_1 (s) - H \bar{X}_2 (s) \cr
    \Longrightarrow &s^2 \bar{X}_1 (s) = D \bar{I} (s) + F \bar{X}_1 (s) - H s \bar{X}_1 (s) \cr
    &D \bar{I} (s) = s^2 \bar{X}_1 (s) + H s \bar{X}_1 (s) - F \bar{X}_1 \cr
    &\bar{I} (s) = \bar{X}_1 (s) \frac{ s^2 + H s - F }{D}
\end{align}
\noindent Find the transfer function, $G_{x}$, with input $\bar{V}$ and output $\bar{X}_1$.
Equations \eqref{a5:laplace_i_dot_0} and \eqref{a5:laplace_i_dot_1} are equal, with all their inputs/outputs in the s-domain: 
\begin{align} \label{a5:tranfer_function}
    \Longrightarrow &\frac{N \bar{V}}{\left( s + P \right)} = \bar{X}_1 \frac{ s^2 + H s - F }{D} \cr
    &D N \bar{V} = \bar{X}_1 \left( s^2 + H s - F \right) \left( s + P \right) \cr
    &D N \bar{V} = \bar{X}_1 \left( s^3 + \left( H + P \right) s^2 + \left( HP - F \right) s - FP \right) \cr
    &G_x = \frac{\bar{X}_1}{\bar{V}} = \frac{D N}{s^3 + \left( H + P \right) s^2 + \left( HP - F \right) s - FP}
\end{align}

\medskip
\noindent Equation \eqref{a5:tranfer_function} shows the transfer function of the system. There are 3 poles in the system. For the system to be oscillatory under impulse response, the poles of the denominator, $s^3 + \left( H + P \right) s^2 + \left( HP - F \right) s - FP$, have to have negative real coefficients.